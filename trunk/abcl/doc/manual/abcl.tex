% -*- mode: latex; -*-
% http://en.wikibooks.org/wiki/LaTeX/
\documentclass[10pt]{book}
\usepackage{abcl}

\begin{document}
\title{A Manual for Armed Bear Common Lisp}
\date{October 2, 2011}
\author{Mark~Evenson, Erik~Huelsmann, Alessio~Stalla, Ville~Voutilainen}

\maketitle

\chapter{Introduction}

Armed Bear is a mostly conforming implementation of the ANSI Common
Lisp standard.  This manual documents the Armed Bear Common Lisp
implementation for users of the system.

\subsection{Version}
This manual corresponds to abcl-0.28.0, as yet unreleased.

\chapter{Running}

ABCL is packaged as a single jar file usually named either
``abcl.jar'' or something like``abcl-0.28.0.jar'' if you are using a
versioned package from your system vendor.  This byte archive can be
executed under the control of a suitable JVM by using the ``-jar''
option to parse the manifest, and select the named class
(org.armedbear.lisp.Main) for excution:

\begin{listing-shell}
  cmd$ java -jar abcl.jar
\end{listing-shell}

N.b. for the proceeding command to work, the ``java'' executable needs
to be in your path.

To make it easier to facilitate the use of ABCL in tool chains (such as
SLIME) the invocation is wrapped in a Bourne shell script under UNIX
or a DOS command script under Windows so that ABCL may be executed
simply as:

\begin{listing-shell}
  cmd$ abcl
\end{listing-shell}

\section{Options}

ABCL supports the following options:

\begin{verbatim}
--help
    Displays this message.
--noinform
    Suppresses the printing of startup information and banner.
--noinit
    Suppresses the loading of the '~/.abclrc' startup file.
--nosystem
    Suppresses loading the 'system.lisp' customization file. 
--eval <FORM>
    Evaluates the <FORM> before initializing REPL.
--load <FILE>
    Loads the file <FILE> before initializing REPL.
--load-system-file <FILE>
    Loads the system file <FILE> before initializing REPL.
--batch
    The process evaluates forms specified by arguments and possibly by those
    by those in the intialization file '~/.abcl', and then exits.

The occurance of '--' copies the remaining arguments, unprocessed, into
the variable EXTENSIONS:*COMMAND-LINE-ARGUMENT-LIST*.
\end{verbatim}

All of the command line arguments which follow the occurrence of ``--''
are passed into a list bound to the EXT:*COMMAND-LINE-ARGUMENT-LIST*
variable.

\section{Initialization}

If the ABCL process is started without the ``--noinit'' flag, it
attempts to load a file named ``.abclrc'' located in the user's home
directory and then interpret its contents.  

The user's home directory is determined by the value of the JVM system
property ``user.home''.

\chapter{Conformance}

\section{ANSI Common Lisp}
ABCL is currently a non-conforming ANSI Common Lisp implementation due
to the following issues:

\begin{itemize}
  \item Missing statement of conformance in accompanying documentation
  \item The generic function signatures of the DOCUMENTATION symbol do
    not match the CLHS.
\end{itemize}

ABCL aims to be be a fully conforming ANSI Common Lisp
implementation.  Any other behavior should be reported as a bug.

\section{Contemporary Common Lisp}
In addition to ANSI conformance, ABCL strives to implement features
expected of a contemporary Common Lisp.
\begin{itemize}
  \item Incomplete (A)MOP 
    % N.B. 
    % TODO go through AMOP with symbols, starting by looking for
    % matching function signature.
    % XXX is this really blocking ANSI conformance?  Answer: we have
    % to start with such a ``census'' to determine what we have.
  \item Incomplete Streams:  need suitable abstraction between ANSI
    and Gray streams.
    
\end{itemize}

\chapter{Interaction with host JVM}

% describe calling Java from Lisp, and calling Lisp from Java,
% probably in two separate sections.  Presumably, we can partition our
% audience into those who are more comfortable with Java, and those
% that are more comforable with Lisp

\section{Lisp to Java}

ABCL offers a number of mechanisms to interact with Java from
its lisp environment. It allows calling methods (and static methods) of
Java objects, manipulation of fields and static fields and construction
of new Java objects.

When calling Java routines, some values will automatically be converted
by the FFI from Lisp values to Java values. These conversions typically
apply to strings, integers and floats. Other values need to be converted
to their Java equivalents by the programmer before calling the Java
object method. Java values returned to Lisp are also generally converted
back to their Lisp counterparts. Some operators make an exception to this
rule and do not perform any conversion; those are the ``raw'' counterparts
of certain FFI functions and are recognizable by their name ending with
\code{-RAW}.

\subsection{Lowlevel Java API}

There's a higher level Java API defined in the
\ref{topic:Higher level Java API: JSS}(JSS package) which is available
in the contrib/ directory. This package is described later in this
document.  This section covers the lower level API directly available
after evaluating \code{(require 'JAVA)}.

\subsubsection{Calling Java object methods}

There are two ways to call a Java object method in the basic API:

\begin{itemize}
\item Call a specific method reference (pre-acquired)
\item Dynamic dispatch using the method name and
  the call-specific arguments provided by finding the
  \ref{section:Parameter matching for FFI dynamic dispatch}{best match}.
\end{itemize}

The dynamic dispatch variant is discussed in the next section.

\code{JAVA:JMETHOD} is used to acquire a specific method reference.
The function takes at two or more arguments. The first is Java class designator
(a \code{JAVA:JAVA-CLASS} object returned by \code{JAVA:JCLASS} or a string naming
a Java class). The second is a string naming the method.

Any arguments beyond the first two should be strings naming Java classes with
one exception as listed in the next paragraph. These
classes specify the types of the arguments for the method to be returned.

There's additional calling convention to the \code{JAVA:JMETHOD} function:
When the method is called with three parameters and the last parameter is an
integer, the first method by that name and matching number of parameters is
returned.

Once you have a reference to the method, you can call it using \code{JAVA:JCALL},
which takes the method as the first argument. The second argument is the
object instance to call the method on, or \code{NIL} in case of a static method.
Any remaining parameters are used as the remaining arguments for the call.

\subsubsection{Calling Java object methods: dynamic dispatch}

The second way of calling Java object methods is by using dynamic dispatch.
In this case \code{JAVA:JCALL} is used directly without acquiring a method
reference first. In this case, the first argument provided to \code{JAVA:JCALL}
is a string naming the method to be called. The second argument is the instance
on which the method should be called and any further arguments are used to
select the best matching method and dispatch the call.

\subsubsection{Dynamic dispatch: caveats}

Dynamic dispatch is performed by using the Java reflection API. Generally
it works fine, but there are corner cases where the API does not correctly
reflect all the details involved in calling a Java method. An example is
the following Java code:

\begin{listing-java}
ZipFile jar = new ZipFile("/path/to/some.jar");
Object els = jar.entries();
Method method = els.getClass().getMethod("hasMoreElements");
method.invoke(els);
\end{listing-java}

even though the method \code{hasMoreElements} is public in \code{Enumeration},
the above code fails with

\begin{listing-java}
java.lang.IllegalAccessException: Class ... can
not access a member of class java.util.zip.ZipFile$2 with modifiers
"public"
       at sun.reflect.Reflection.ensureMemberAccess(Reflection.java:65)
       at java.lang.reflect.Method.invoke(Method.java:583)
       at ...
\end{listing-java}

because the method has been overridden by a non-public class and the
reflection API, unlike javac, is not able to handle such a case.

While code like that is uncommon in Java, it is typical of ABCL's FFI
calls. The code above corresponds to the following Lisp code:

\begin{listing-lisp}
(let ((jar (jnew "java.util.zip.ZipFile" "/path/to/some.jar")))
  (let ((els (jcall "entries" jar)))
    (jcall "hasMoreElements" els)))
\end{listing-lisp}

except that the dynamic dispatch part is not shown.

To avoid such pitfalls, all Java objects in ABCL carry an extra
field representing the ``intended class'' of the object. That is the class
that is used first by \code{JAVA:JCALL} and similar to resolve methods;
the actual class of the object is only tried if the method is not found
in the intended class. Of course, the intended class is always a superclass
of the actual class - in the worst case, they coincide. The intended class
is deduced by the return type of the method that originally returned
the Java object; in the case above, the intended class of \code{ELS}
is \code{java.util.Enumeration} because that's the return type of
the \code{entries} method.

While this strategy is generally effective, there are cases where the
intended class becomes too broad to be useful. The typical example
is the extraction of an element from a collection, since methods in
the collection API erase all types to \code{Object}. The user can
always force a more specific intended class by using the \code{JAVA:JCOERCE}
operator.

% \begin{itemize}
% \item Java values are accessible as objects of type JAVA:JAVA-OBJECT.
% \item The Java FFI presents a Lisp package (JAVA) with many useful
%   symbols for manipulating the artifacts of expectation on the JVM,
%   including creation of new objects \ref{JAVA:JNEW}, \ref{JAVA:JMETHOD}), the
%   introspection of values \ref{JAVA:JFIELD}, the execution of methods
%   (\ref{JAVA:JCALL}, \ref{JAVA:JCALL-RAW}, \ref{JAVA:JSTATIC})
% \item The JSS package (\ref{JSS}) in contrib introduces a convenient macro
%   syntax \ref{JSS:SHARPSIGN_DOUBLEQUOTE_MACRO} for accessing Java
%   methods, and additional convenience functions.
% \item Java classes and libraries may be dynamically added to the
%   classpath at runtime (JAVA:ADD-TO-CLASSPATH).
% \end{itemize}

\subsubsection{Calling Java class static methods}

Like with non-static methods, references to static methods can be acquired
by using the \code{JAVA:JMETHOD} primitive. In order to call this method,
it's not possible to use the \code{JAVA:JCALL} primitive however: there's a 
separate API to retrieve a reference to static methods. This
primitive is called \code{JAVA:JSTATIC}. 

Like \code{JAVA:JCALL}, \code{JAVA:JSTATIC} supports dynamic dispatch by
passing the name of the method as a string instead of passing a method reference.
The parameter values should be values to pass in the function call instead of
a specification of classes for each parameter.

\subsubsection{Parameter matching for FFI dynamic dispatch}

The algorithm used to resolve the best matching method given the name
and the arguments' types is the same as described in the Java Language
Specification. Any deviation should be reported as a bug.

% ###TODO reference to correct JLS section

\subsubsection{Instantiating Java objects}

Java objects can be instantiated (created) from Lisp by calling
a constructor from the class of the object to be created. The same way
\code{JAVA:JMETHOD} is used to acquire a method reference, the
\code{JAVA:JCONSTRUCTOR} primitive can be used to acquire a constructor
reference. It's arguments specify the types of arguments of the constructor
method the same way as with \code{JAVA:JMETHOD}.

The constructor can't be passed to \code{JAVA:JCALL}, but instead should
be passed as an argument to \code{JAVA:JNEW}.

\section{Lisp from Java}

In order to access the Lisp world from Java, one needs to be aware
of a few things. The most important ones are listed below.

\begin{itemize}
\item All Lisp values are descendants of LispObject.java
\item In order to 
\item Lisp symbols are accessible via either directly referencing the
  Symbol.java instance or by dynamically introspecting the
  corresponding Package.java instance.
\item The Lisp dynamic environment may be saved via
  \code{LispThread.bindSpecial(Binding)} and restored via
  \code{LispThread.resetSpecialBindings(Mark)}.
\item Functions may be executed by invocation of the
  Function.execute(args [...]) 
\end{itemize}

\subsection{Lisp FFI}

FFI stands for "Foreign Function Interface" which is the phase which
the contemporary Lisp world refers to methods of "calling out" from
Lisp into "foreign" languages and environments.  This document
describes the various ways that one interacts with Lisp world of ABCL
from Java, considering the hosted Lisp as the "Foreign Function" that
needs to be "Interfaced".

\subsection{Calling Lisp from Java}

Note: As the entire ABCL Lisp system resides in the org.armedbear.lisp
package the following code snippets do not show the relevant import
statements in the interest of brevity.  An example of the import
statement would be

\begin{listing-java}
  import org.armedbear.lisp.*;
\end{listing-java}

to potentially import all the JVM symbol from the `org.armedbear.lisp'
namespace.

Per JVM, there can only ever be a single Lisp interpreter.  This is
started by calling the static method `Interpreter.createInstance()`.

\begin{listing-java}
  Interpreter interpreter = Interpreter.createInstance();
\end{listing-java}

If this method has already been invoked in the lifetime of the current
Java process it will return null, so if you are writing Java whose
life-cycle is a bit out of your control (like in a Java servlet), a
safer invocation pattern might be:

\begin{listing-java}
  Interpreter interpreter = Interpreter.getInstance();
  if (interpreter == null) {
    interpreter = Interpreter.createInstance();
  }
\end{listing-java}


The Lisp \code{eval} primitive may be simply passed strings for evaluation,
as follows

\begin{listing-java}
  String line = "(load \"file.lisp\")";
  LispObject result = interpreter.eval(line);
\end{listing-java}

Notice that all possible return values from an arbitrary Lisp
computation are collapsed into a single return value.  Doing useful
further computation on the ``LispObject'' depends on knowing what the
result of the computation might be, usually involves some amount
of \code{instanceof} introspection, and forms a whole topic to itself
(c.f. [Introspecting a LispObject])

Using \code{eval} involves the Lisp interpreter.  Lisp functions may
be directly invoked by Java method calls as follows.  One simply
locates the package containing the symbol, then obtains a reference to
the symbol, and then invokes the \code{execute()} method with the
desired parameters.

\begin{listing-java}
    interpreter.eval("(defun foo (msg) (format nil \"You told me '~A'~%\" msg))");
    Package pkg = Packages.findPackage("CL-USER");
    Symbol foo = pkg.findAccessibleSymbol("FOO"); 
    Function fooFunction = (Function)foo.getSymbolFunction();
    JavaObject parameter = new JavaObject("Lisp is fun!");
    LispObject result = fooFunction.execute(parameter);
    // How to get the "naked string value"?
    System.out.println("The result was " + result.writeToString()); 
\end{listing-java}

If one is calling an primitive function in the CL package the syntax
becomes considerably simpler.  If we can locate the instance of
definition in the ABCL Java source, we can invoke the symbol directly.
For instnace, to tell if a `LispObject` contains a reference to a symbol.

\begin{listing-java}
    boolean nullp(LispObject object) {
      LispObject result = Primitives.NULL.execute(object);
      if (result == NIL) { // the symbol 'NIL' is explicity named in the Java
                           // namespace at ``Symbol.NIL''
                           // but is always present in the
                           // localnamespace in its unadorned form for
                           // the convenience of the User.
        return false;
      }
      return true;
   }
\end{listing-java}

\subsubsection{Introspecting a LispObject}
\label{topic:Introspecting a LispObject}

We present various patterns for introspecting an an arbitrary
`LispObject` which can represent the result of every Lisp evaluation
into semantics that Java can meaningfully deal with.

\subsubsection{LispObject as \code{boolean}}

If the LispObject a generalized boolean values, one can use
\code{getBooleanValue()} to convert to Java:

\begin{listing-java}
     LispObject object = Symbol.NIL;
     boolean javaValue = object.getBooleanValue();
\end{listing-java}

Although since in Lisp, any value other than NIL means "true"
(so-called generalized Boolean), the use of Java equality it quite a
bit easier to type and more optimal in terms of information it conveys
to the compiler would be:

\begin{listing-java}
    boolean javaValue = (object != Symbol.NIL);
\end{listing-java}

\paragraph{LispObject is a list}

If LispObject is a list, it will have the type `Cons`.  One can then use
the \code{copyToArray} to make things a bit more suitable for Java
iteration.

\begin{listing-java}
    LispObject result = interpreter.eval("'(1 2 4 5)");
    if (result instanceof Cons) {
      LispObject array[] = ((Cons)result.copyToArray());
      ...
    }
\end{listing-java}

A more Lispy way to iterated down a list is to use the `cdr()` access
function just as like one would traverse a list in Lisp:;

\begin{listing-java}
    LispObject result = interpreter.eval("'(1 2 4 5)");
    while (result != Symbol.NIL) {
      doSomething(result.car());
      result = result.cdr();
    }
\end{listing-java}

\subsection{Java Scripting API (JSR-223)}

ABCL can be built with support for JSR-223, which offers a language-agnostic
API to invoke other languages from Java. The binary distribution downloadable
from ABCL's common-lisp.net home is built with JSR-223 support. If you're building
ABCL from source on a pre-1.6 JVM, you need to have a JSR-223 implementation in your
CLASSPATH (such as Apache Commons BSF 3.x or greater) in order to build ABCL
with JSR-223 support; otherwise, this feature will not be built.

This section describes the design decisions behind the ABCL JSR-223 support. It is not a description of what JSR-223 is or a tutorial on how to use it. See http://trac.common-lisp.net/armedbear/browser/trunk/abcl/examples/jsr-223 for example usage.

\subsubsection{Conversions}

In general, ABCL's implementation of the JSR-223 API performs implicit conversion from Java objects to Lisp objects when invoking Lisp from Java, and the opposite when returning values from Java to Lisp. This potentially reduces coupling between user code and ABCL. To avoid such conversions, wrap the relevant objects in \code{JavaObject} instances.

\subsubsection{Implemented JSR-223 interfaces}

JSR-223 defines three main interfaces, of which two (Invocable and Compilable) are optional. ABCL implements all the three interfaces - ScriptEngine and the two optional ones - almost completely. While the JSR-223 API is not specific to a single scripting language, it was designed with languages with a more or less Java-like object model in mind: languages such as Javascript, Python, Ruby, which have a concept of "class" or "object" with "fields" and "methods". Lisp is a bit different, so certain adaptations were made, and in one case a method has been left unimplemented since it does not map at all to Lisp.

\subsubsection{The ScriptEngine}

The main interface defined by JSR-223, javax.script.ScriptEngine, is implemented by the class \code{org.armedbear.lisp.scripting.AbclScriptEngine}. AbclScriptEngine is a singleton, reflecting the fact that ABCL is a singleton as well. You can obtain an instance of AbclScriptEngine using the  AbclScriptEngineFactory or by using the service provider mechanism through ScriptEngineManager (refer to the javax.script documentation).

\subsubsection{Startup and configuration file}

At startup (i.e. when its constructor is invoked, as part of the static initialization phase of AbclScriptEngineFactory) the ABCL script engine attempts to load an "init file" from the classpath (/abcl-script-config.lisp). If present, this file can be used to customize the behaviour of the engine, by setting a number of variables in the ABCL-SCRIPT package. Here is a list of the available variables:

\begin{itemize}
\item *use-throwing-debugger* Controls whether ABCL uses a non-standard debugging hook function to throw a Java exception instead of dropping into the debugger in case of unhandled error conditions.
  \begin{itemize}
  \item Default value: T
  \item Rationale: it is more convenient for Java programmers using Lisp as a scripting language to have it return exceptions to Java instead of handling them in the Lisp world.
  \item Known Issues: the non-standard debugger hook has been reported to misbehave in certain circumstances, so consider disabling it if it doesn't work for you.
  \end{itemize}
\item *launch-swank-at-startup* If true, Swank will be launched at startup. See *swank-dir* and *swank-port*.
  \begin{itemize}
  \item Default value: NIL
  \end{itemize}
\item *swank-dir* The directory where Swank is installed. Must be set if *launch-swank-at-startup* is true.
\item *swank-port* The port where Swank will listen for connections. Must be set if *launch-swank-at-startup* is true.
  \begin{itemize}
  \item Default value: 4005
  \end{itemize}
\end{itemize}

Additionally, at startup the AbclScriptEngine will \code{(require 'asdf)} - in fact, it uses asdf to load Swank.

\subsubsection{Evaluation}

Code is read and evaluated in the package ABCL-SCRIPT-USER. This packages USEs the COMMON-LISP, JAVA and ABCL-SCRIPT packages. Future versions of the script engine might make this default package configurable. The \code{CL:LOAD} function is used under the hood for evaluating code, and thus the same behavior of LOAD is guaranteed. This allows, among other things, \code{IN-PACKAGE} forms to change the package in which the loaded code is read.

It is possible to evaluate code in what JSR-223 calls a "ScriptContext" (basically a flat environment of name->value pairs). This context is used to establish special bindings for all the variables defined in it; since variable names are strings from Java's point of view, they are first interned using READ-FROM-STRING with, as usual, ABCL-SCRIPT-USER as the default package. Variables are declared special because CL's \code{LOAD}, \code{EVAL} and \code{COMPILE} functions work in a null lexical environment and would ignore non-special bindings.

Contrary to what the function \code{LOAD} does, evaluation of a series of forms returns the value of the last form instead of T, so the evaluation of short scripts does the Right Thing.

\subsubsection{Compilation}

AbclScriptEngine implements the javax.script.Compilable interface. Currently it only supports compilation using temporary files. Compiled code, returned as an instance of javax.script.CompiledScript, is read, compiled and executed by default in the ABCL-SCRIPT-USER package, just like evaluated code. Differently from evaluated code, though, due to the way the ABCL compiler works, compiled code contains no reference to top-level self-evaluating objects (like numbers or strings). Thus, when evaluated, a piece of compiled code will return the value of the last non-self-evaluating form: for example the code "(do-something) 42" will return 42 when interpreted, but will return the result of (do-something) when compiled and later evaluated. To ensure consistency of behavior between interpreted and compiled code, make sure the last form is always a compound form - at least (identity some-literal-object). Note that this issue should not matter in real code, where it is unlikely a top-level self-evaluating form will appear as the last form in a file (in fact, the Common Lisp load function always returns T upon success; with JSR-223 this policy has been changed to make evaluation of small code snippets work as intended).

\subsubsection{Invocation of functions and methods}

AbclScriptEngine implements the \code{javax.script.Invocable} interface, which allows to directly call Lisp functions and methods, and to obtain Lisp implementations of Java interfaces. This is only partially possible with Lisp since it has functions, but not methods - not in the traditional OO sense, at least, since Lisp methods are not attached to objects but belong to generic functions. Thus, the method \code{invokeMethod()} is not implemented and throws an UnsupportedOperationException when called. The \code{invokeFunction()} method should be used to call both regular and generic functions.

\subsubsection{Implementation of Java interfaces in Lisp}

ABCL can use the Java reflection-based proxy feature to implement Java interfaces in Lisp. It has several built-in ways to implement an interface, and supports definition of new ones. The \code{JAVA:JMAKE-PROXY} generic function is used to make such proxies. It has the following signature:

\code{jmake-proxy interface implementation \&optional lisp-this ==> proxy}

\code{interface} is a Java interface metaobject (e.g. obtained by invoking \code{jclass}) or a string naming a Java interface. \code{implementation} is the object used to implement the interface - several built-in methods of jmake-proxy exist for various types of implementations. \code{lisp-this} is an object passed to the closures implementing the Lisp "methods" of the interface, and defaults to \code{NIL}.

The returned proxy is an instance of the interface, with methods implemented with Lisp functions.

Built-in interface-implementation types include:

\begin{itemize}
\item a single Lisp function which upon invocation of any method in the interface will be passed the method name, the Lisp-this object, and all the parameters. Useful for interfaces with a single method, or to implement custom interface-implementation strategies.
\item a hash-map of method-name -> Lisp function mappings. Function signature is \code{(lisp-this \&rest args)}.
\item a Lisp package. The name of the Java method to invoke is first transformed in an idiomatic Lisp name (\code{javaMethodName} becomes \code{JAVA-METHOD-NAME}) and a symbol with that name is searched in the package. If it exists and is fbound, the corresponding function will be called. Function signature is as the hash-table case.
\end{itemize}

This functionality is exposed by the AbclScriptEngine with the two methods getInterface(Class) and getInterface(Object, Class). The former returns an interface implemented with the current Lisp package, the latter allows the programmer to pass an interface-implementation object which will in turn be passed to the jmake-proxy generic function.

\chapter{Implementation Dependent Extensions}

As outlined by the CLHS ANSI conformance guidelines, we document the
extensions to the Armed Bear Lisp implementation made accessible to
the user by virtue of being an exported symbol in the JAVA, THREADS,
or EXTENSIONS packages.

\section{JAVA}

\subsection{Modifying the JVM CLASSPATH}

The JAVA:ADD-TO-CLASSPATH generic functions allows one to add the
specified pathname or list of pathnames to the current classpath
used by ABCL, allowing the dynamic loading of JVM objects:

\begin{listing-lisp}
CL-USER> (add-to-classpath "/path/to/some.jar")
\end{listing-lisp}

NB \code{add-to-classpath} only affects the classloader used by ABCL
(the value of the special variable \code{JAVA:*CLASSLOADER*}. It has
no effect on Java code outside ABCL.

\subsection{API}

% include autogen docs for the JAVA package.
\paragraph{}
\label{JAVA:*JAVA-OBJECT-TO-STRING-LENGTH*}
\index{*JAVA-OBJECT-TO-STRING-LENGTH*}
--- Variable: \textbf{*java-object-to-string-length*} [\textbf{java}] \textit{}

\begin{adjustwidth}{5em}{5em}
Length to truncate toString() PRINT-OBJECT output for an otherwise unspecialized JAVA-OBJECT.  Can be set to NIL to indicate no limit.
\end{adjustwidth}

\paragraph{}
\label{JAVA:+FALSE+}
\index{+FALSE+}
--- Variable: \textbf{+false+} [\textbf{java}] \textit{}

\begin{adjustwidth}{5em}{5em}
The JVM primitive value for boolean false.
\end{adjustwidth}

\paragraph{}
\label{JAVA:+NULL+}
\index{+NULL+}
--- Variable: \textbf{+null+} [\textbf{java}] \textit{}

\begin{adjustwidth}{5em}{5em}
The JVM null object reference.
\end{adjustwidth}

\paragraph{}
\label{JAVA:+TRUE+}
\index{+TRUE+}
--- Variable: \textbf{+true+} [\textbf{java}] \textit{}

\begin{adjustwidth}{5em}{5em}
The JVM primitive value for boolean true.
\end{adjustwidth}

\paragraph{}
\label{JAVA:ADD-TO-CLASSPATH}
\index{ADD-TO-CLASSPATH}
--- Generic Function: \textbf{add-to-classpath} [\textbf{java}] \textit{}

\begin{adjustwidth}{5em}{5em}
Add JAR-OR-JARS to the JVM classpath optionally specifying the CLASSLOADER to add.

JAR-OR-JARS is either a pathname designating a jar archive or the root
directory to search for classes or a list of such values.
\end{adjustwidth}

\paragraph{}
\label{JAVA:CHAIN}
\index{CHAIN}
--- Macro: \textbf{chain} [\textbf{java}] \textit{}

\begin{adjustwidth}{5em}{5em}
not-documented
\end{adjustwidth}

\paragraph{}
\label{JAVA:DEFINE-JAVA-CLASS}
\index{DEFINE-JAVA-CLASS}
--- Macro: \textbf{define-java-class} [\textbf{java}] \textit{}

\begin{adjustwidth}{5em}{5em}
not-documented
\end{adjustwidth}

\paragraph{}
\label{JAVA:DESCRIBE-JAVA-OBJECT}
\index{DESCRIBE-JAVA-OBJECT}
--- Function: \textbf{describe-java-object} [\textbf{java}] \textit{}

\begin{adjustwidth}{5em}{5em}
not-documented
\end{adjustwidth}

\paragraph{}
\label{JAVA:DUMP-CLASSPATH}
\index{DUMP-CLASSPATH}
--- Function: \textbf{dump-classpath} [\textbf{java}] \textit{\&optional classloader}

\begin{adjustwidth}{5em}{5em}
not-documented
\end{adjustwidth}

\paragraph{}
\label{JAVA:ENSURE-JAVA-CLASS}
\index{ENSURE-JAVA-CLASS}
--- Function: \textbf{ensure-java-class} [\textbf{java}] \textit{jclass}

\begin{adjustwidth}{5em}{5em}
Attempt to ensure that the Java class referenced by JCLASS exists in the current process of the implementation.
\end{adjustwidth}

\paragraph{}
\label{JAVA:ENSURE-JAVA-OBJECT}
\index{ENSURE-JAVA-OBJECT}
--- Function: \textbf{ensure-java-object} [\textbf{java}] \textit{obj}

\begin{adjustwidth}{5em}{5em}
Ensures OBJ is wrapped in a JAVA-OBJECT, wrapping it if necessary.
\end{adjustwidth}

\paragraph{}
\label{JAVA:GET-CURRENT-CLASSLOADER}
\index{GET-CURRENT-CLASSLOADER}
--- Function: \textbf{get-current-classloader} [\textbf{java}] \textit{}

\begin{adjustwidth}{5em}{5em}
not-documented
\end{adjustwidth}

\paragraph{}
\label{JAVA:GET-DEFAULT-CLASSLOADER}
\index{GET-DEFAULT-CLASSLOADER}
--- Function: \textbf{get-default-classloader} [\textbf{java}] \textit{}

\begin{adjustwidth}{5em}{5em}
not-documented
\end{adjustwidth}

\paragraph{}
\label{JAVA:JARRAY-COMPONENT-TYPE}
\index{JARRAY-COMPONENT-TYPE}
--- Function: \textbf{jarray-component-type} [\textbf{java}] \textit{atype}

\begin{adjustwidth}{5em}{5em}
Returns the component type of the array type ATYPE
\end{adjustwidth}

\paragraph{}
\label{JAVA:JARRAY-FROM-LIST}
\index{JARRAY-FROM-LIST}
--- Function: \textbf{jarray-from-list} [\textbf{java}] \textit{list}

\begin{adjustwidth}{5em}{5em}
Return a Java array from LIST whose type is inferred from the first element.

For more control over the type of the array, use JNEW-ARRAY-FROM-LIST.
\end{adjustwidth}

\paragraph{}
\label{JAVA:JARRAY-LENGTH}
\index{JARRAY-LENGTH}
--- Function: \textbf{jarray-length} [\textbf{java}] \textit{java-array}

\begin{adjustwidth}{5em}{5em}
Returns the length of a Java primitive array.
\end{adjustwidth}

\paragraph{}
\label{JAVA:JARRAY-REF}
\index{JARRAY-REF}
--- Function: \textbf{jarray-ref} [\textbf{java}] \textit{java-array \&rest indices}

\begin{adjustwidth}{5em}{5em}
Dereferences the Java array JAVA-ARRAY using the given INDICIES, coercing the result into a Lisp object, if possible.
\end{adjustwidth}

\paragraph{}
\label{JAVA:JARRAY-REF-RAW}
\index{JARRAY-REF-RAW}
--- Function: \textbf{jarray-ref-raw} [\textbf{java}] \textit{java-array \&rest indices}

\begin{adjustwidth}{5em}{5em}
Dereference the Java array JAVA-ARRAY using the given INDICIES. Does not attempt to coerce the result into a Lisp object.
\end{adjustwidth}

\paragraph{}
\label{JAVA:JARRAY-SET}
\index{JARRAY-SET}
--- Function: \textbf{jarray-set} [\textbf{java}] \textit{java-array new-value \&rest indices}

\begin{adjustwidth}{5em}{5em}
Stores NEW-VALUE at the given index in JAVA-ARRAY.
\end{adjustwidth}

\paragraph{}
\label{JAVA:JAVA-CLASS}
\index{JAVA-CLASS}
--- Class: \textbf{java-class} [\textbf{java}] \textit{}

\begin{adjustwidth}{5em}{5em}
not-documented
\end{adjustwidth}

\paragraph{}
\label{JAVA:JAVA-EXCEPTION}
\index{JAVA-EXCEPTION}
--- Class: \textbf{java-exception} [\textbf{java}] \textit{}

\begin{adjustwidth}{5em}{5em}
not-documented
\end{adjustwidth}

\paragraph{}
\label{JAVA:JAVA-EXCEPTION-CAUSE}
\index{JAVA-EXCEPTION-CAUSE}
--- Function: \textbf{java-exception-cause} [\textbf{java}] \textit{java-exception}

\begin{adjustwidth}{5em}{5em}
not-documented
\end{adjustwidth}

\paragraph{}
\label{JAVA:JAVA-OBJECT}
\index{JAVA-OBJECT}
--- Class: \textbf{java-object} [\textbf{java}] \textit{}

\begin{adjustwidth}{5em}{5em}
not-documented
\end{adjustwidth}

\paragraph{}
\label{JAVA:JAVA-OBJECT-P}
\index{JAVA-OBJECT-P}
--- Function: \textbf{java-object-p} [\textbf{java}] \textit{object}

\begin{adjustwidth}{5em}{5em}
Returns T if OBJECT is a JAVA-OBJECT.
\end{adjustwidth}

\paragraph{}
\label{JAVA:JCALL}
\index{JCALL}
--- Function: \textbf{jcall} [\textbf{java}] \textit{method-ref instance \&rest args}

\begin{adjustwidth}{5em}{5em}
Invokes the Java method METHOD-REF on INSTANCE with arguments ARGS, coercing the result into a Lisp object, if possible.
\end{adjustwidth}

\paragraph{}
\label{JAVA:JCALL-RAW}
\index{JCALL-RAW}
--- Function: \textbf{jcall-raw} [\textbf{java}] \textit{method-ref instance \&rest args}

\begin{adjustwidth}{5em}{5em}
Invokes the Java method METHOD-REF on INSTANCE with arguments ARGS. Does not attempt to coerce the result into a Lisp object.
\end{adjustwidth}

\paragraph{}
\label{JAVA:JCLASS}
\index{JCLASS}
--- Function: \textbf{jclass} [\textbf{java}] \textit{name-or-class-ref \&optional class-loader}

\begin{adjustwidth}{5em}{5em}
Returns a reference to the Java class designated by NAME-OR-CLASS-REF. If the CLASS-LOADER parameter is passed, the class is resolved with respect to the given ClassLoader.
\end{adjustwidth}

\paragraph{}
\label{JAVA:JCLASS-ARRAY-P}
\index{JCLASS-ARRAY-P}
--- Function: \textbf{jclass-array-p} [\textbf{java}] \textit{class}

\begin{adjustwidth}{5em}{5em}
Returns T if CLASS is an array class
\end{adjustwidth}

\paragraph{}
\label{JAVA:JCLASS-CONSTRUCTORS}
\index{JCLASS-CONSTRUCTORS}
--- Function: \textbf{jclass-constructors} [\textbf{java}] \textit{class}

\begin{adjustwidth}{5em}{5em}
Returns a vector of constructors for CLASS
\end{adjustwidth}

\paragraph{}
\label{JAVA:JCLASS-FIELD}
\index{JCLASS-FIELD}
--- Function: \textbf{jclass-field} [\textbf{java}] \textit{class field-name}

\begin{adjustwidth}{5em}{5em}
Returns the field named FIELD-NAME of CLASS
\end{adjustwidth}

\paragraph{}
\label{JAVA:JCLASS-FIELDS}
\index{JCLASS-FIELDS}
--- Function: \textbf{jclass-fields} [\textbf{java}] \textit{class \&key declared public}

\begin{adjustwidth}{5em}{5em}
Returns a vector of all (or just the declared/public, if DECLARED/PUBLIC is true) fields of CLASS
\end{adjustwidth}

\paragraph{}
\label{JAVA:JCLASS-INTERFACE-P}
\index{JCLASS-INTERFACE-P}
--- Function: \textbf{jclass-interface-p} [\textbf{java}] \textit{class}

\begin{adjustwidth}{5em}{5em}
Returns T if CLASS is an interface
\end{adjustwidth}

\paragraph{}
\label{JAVA:JCLASS-INTERFACES}
\index{JCLASS-INTERFACES}
--- Function: \textbf{jclass-interfaces} [\textbf{java}] \textit{class}

\begin{adjustwidth}{5em}{5em}
Returns the vector of interfaces of CLASS
\end{adjustwidth}

\paragraph{}
\label{JAVA:JCLASS-METHODS}
\index{JCLASS-METHODS}
--- Function: \textbf{jclass-methods} [\textbf{java}] \textit{class \&key declared public}

\begin{adjustwidth}{5em}{5em}
Return a vector of all (or just the declared/public, if DECLARED/PUBLIC is true) methods of CLASS
\end{adjustwidth}

\paragraph{}
\label{JAVA:JCLASS-NAME}
\index{JCLASS-NAME}
--- Function: \textbf{jclass-name} [\textbf{java}] \textit{class-ref \&optional name}

\begin{adjustwidth}{5em}{5em}
When called with one argument, returns the name of the Java class
  designated by CLASS-REF. When called with two arguments, tests
  whether CLASS-REF matches NAME.
\end{adjustwidth}

\paragraph{}
\label{JAVA:JCLASS-OF}
\index{JCLASS-OF}
--- Function: \textbf{jclass-of} [\textbf{java}] \textit{object \&optional name}

\begin{adjustwidth}{5em}{5em}
Returns the name of the Java class of OBJECT. If the NAME argument is
  supplied, verifies that OBJECT is an instance of the named class. The name
  of the class or nil is always returned as a second value.
\end{adjustwidth}

\paragraph{}
\label{JAVA:JCLASS-SUPERCLASS}
\index{JCLASS-SUPERCLASS}
--- Function: \textbf{jclass-superclass} [\textbf{java}] \textit{class}

\begin{adjustwidth}{5em}{5em}
Returns the superclass of CLASS, or NIL if it hasn't got one
\end{adjustwidth}

\paragraph{}
\label{JAVA:JCLASS-SUPERCLASS-P}
\index{JCLASS-SUPERCLASS-P}
--- Function: \textbf{jclass-superclass-p} [\textbf{java}] \textit{class-1 class-2}

\begin{adjustwidth}{5em}{5em}
Returns T if CLASS-1 is a superclass or interface of CLASS-2
\end{adjustwidth}

\paragraph{}
\label{JAVA:JCOERCE}
\index{JCOERCE}
--- Function: \textbf{jcoerce} [\textbf{java}] \textit{object intended-class}

\begin{adjustwidth}{5em}{5em}
Attempts to coerce OBJECT into a JavaObject of class INTENDED-CLASS.  Raises a TYPE-ERROR if no conversion is possible.
\end{adjustwidth}

\paragraph{}
\label{JAVA:JCONSTRUCTOR}
\index{JCONSTRUCTOR}
--- Function: \textbf{jconstructor} [\textbf{java}] \textit{class-ref \&rest parameter-class-refs}

\begin{adjustwidth}{5em}{5em}
Returns a reference to the Java constructor of CLASS-REF with the given PARAMETER-CLASS-REFS.
\end{adjustwidth}

\paragraph{}
\label{JAVA:JCONSTRUCTOR-PARAMS}
\index{JCONSTRUCTOR-PARAMS}
--- Function: \textbf{jconstructor-params} [\textbf{java}] \textit{constructor}

\begin{adjustwidth}{5em}{5em}
Returns a vector of parameter types (Java classes) for CONSTRUCTOR
\end{adjustwidth}

\paragraph{}
\label{JAVA:JEQUAL}
\index{JEQUAL}
--- Function: \textbf{jequal} [\textbf{java}] \textit{obj1 obj2}

\begin{adjustwidth}{5em}{5em}
Compares obj1 with obj2 using java.lang.Object.equals()
\end{adjustwidth}

\paragraph{}
\label{JAVA:JFIELD}
\index{JFIELD}
--- Function: \textbf{jfield} [\textbf{java}] \textit{class-ref-or-field field-or-instance \&optional instance value}

\begin{adjustwidth}{5em}{5em}
Retrieves or modifies a field in a Java class or instance.

Supported argument patterns:

   Case 1: class-ref  field-name:
      Retrieves the value of a static field.

   Case 2: class-ref  field-name  instance-ref:
      Retrieves the value of a class field of the instance.

   Case 3: class-ref  field-name  primitive-value:
      Stores a primitive-value in a static field.

   Case 4: class-ref  field-name  instance-ref  value:
      Stores value in a class field of the instance.

   Case 5: class-ref  field-name  nil  value:
      Stores value in a static field (when value may be
      confused with an instance-ref).

   Case 6: field-name  instance:
      Retrieves the value of a field of the instance. The
      class is derived from the instance.

   Case 7: field-name  instance  value:
      Stores value in a field of the instance. The class is
      derived from the instance.


\end{adjustwidth}

\paragraph{}
\label{JAVA:JFIELD-NAME}
\index{JFIELD-NAME}
--- Function: \textbf{jfield-name} [\textbf{java}] \textit{field}

\begin{adjustwidth}{5em}{5em}
Returns the name of FIELD as a Lisp string
\end{adjustwidth}

\paragraph{}
\label{JAVA:JFIELD-RAW}
\index{JFIELD-RAW}
--- Function: \textbf{jfield-raw} [\textbf{java}] \textit{class-ref-or-field field-or-instance \&optional instance value}

\begin{adjustwidth}{5em}{5em}
Retrieves or modifies a field in a Java class or instance. Does not
attempt to coerce its value or the result into a Lisp object.

Supported argument patterns:

   Case 1: class-ref  field-name:
      Retrieves the value of a static field.

   Case 2: class-ref  field-name  instance-ref:
      Retrieves the value of a class field of the instance.

   Case 3: class-ref  field-name  primitive-value:
      Stores a primitive-value in a static field.

   Case 4: class-ref  field-name  instance-ref  value:
      Stores value in a class field of the instance.

   Case 5: class-ref  field-name  nil  value:
      Stores value in a static field (when value may be
      confused with an instance-ref).

   Case 6: field-name  instance:
      Retrieves the value of a field of the instance. The
      class is derived from the instance.

   Case 7: field-name  instance  value:
      Stores value in a field of the instance. The class is
      derived from the instance.


\end{adjustwidth}

\paragraph{}
\label{JAVA:JFIELD-TYPE}
\index{JFIELD-TYPE}
--- Function: \textbf{jfield-type} [\textbf{java}] \textit{field}

\begin{adjustwidth}{5em}{5em}
Returns the type (Java class) of FIELD
\end{adjustwidth}

\paragraph{}
\label{JAVA:JINPUT-STREAM}
\index{JINPUT-STREAM}
--- Function: \textbf{jinput-stream} [\textbf{java}] \textit{pathname}

\begin{adjustwidth}{5em}{5em}
Returns a java.io.InputStream for resource denoted by PATHNAME.
\end{adjustwidth}

\paragraph{}
\label{JAVA:JINSTANCE-OF-P}
\index{JINSTANCE-OF-P}
--- Function: \textbf{jinstance-of-p} [\textbf{java}] \textit{obj class}

\begin{adjustwidth}{5em}{5em}
OBJ is an instance of CLASS (or one of its subclasses)
\end{adjustwidth}

\paragraph{}
\label{JAVA:JINTERFACE-IMPLEMENTATION}
\index{JINTERFACE-IMPLEMENTATION}
--- Function: \textbf{jinterface-implementation} [\textbf{java}] \textit{interface \&rest method-names-and-defs}

\begin{adjustwidth}{5em}{5em}
Creates and returns an implementation of a Java interface with
   methods calling Lisp closures as given in METHOD-NAMES-AND-DEFS.

   INTERFACE is either a Java interface or a string naming one.

   METHOD-NAMES-AND-DEFS is an alternating list of method names
   (strings) and method definitions (closures).

   For missing methods, a dummy implementation is provided that
   returns nothing or null depending on whether the return type is
   void or not. This is for convenience only, and a warning is issued
   for each undefined method.
\end{adjustwidth}

\paragraph{}
\label{JAVA:JMAKE-INVOCATION-HANDLER}
\index{JMAKE-INVOCATION-HANDLER}
--- Function: \textbf{jmake-invocation-handler} [\textbf{java}] \textit{function}

\begin{adjustwidth}{5em}{5em}
not-documented
\end{adjustwidth}

\paragraph{}
\label{JAVA:JMAKE-PROXY}
\index{JMAKE-PROXY}
--- Generic Function: \textbf{jmake-proxy} [\textbf{java}] \textit{}

\begin{adjustwidth}{5em}{5em}
Returns a proxy Java object implementing the provided interface(s) using methods implemented in Lisp - typically closures, but implementations are free to provide other mechanisms. You can pass an optional 'lisp-this' object that will be passed to the implementing methods as their first argument. If you don't provide this object, NIL will be used. The second argument of the Lisp methods is the name of the Java method being implemented. This has the implication that overloaded methods are merged, so you have to manually discriminate them if you want to. The remaining arguments are java-objects wrapping the method's parameters.
\end{adjustwidth}

\paragraph{}
\label{JAVA:JMEMBER-PROTECTED-P}
\index{JMEMBER-PROTECTED-P}
--- Function: \textbf{jmember-protected-p} [\textbf{java}] \textit{member}

\begin{adjustwidth}{5em}{5em}
MEMBER is a protected member of its declaring class
\end{adjustwidth}

\paragraph{}
\label{JAVA:JMEMBER-PUBLIC-P}
\index{JMEMBER-PUBLIC-P}
--- Function: \textbf{jmember-public-p} [\textbf{java}] \textit{member}

\begin{adjustwidth}{5em}{5em}
MEMBER is a public member of its declaring class
\end{adjustwidth}

\paragraph{}
\label{JAVA:JMEMBER-STATIC-P}
\index{JMEMBER-STATIC-P}
--- Function: \textbf{jmember-static-p} [\textbf{java}] \textit{member}

\begin{adjustwidth}{5em}{5em}
MEMBER is a static member of its declaring class
\end{adjustwidth}

\paragraph{}
\label{JAVA:JMETHOD}
\index{JMETHOD}
--- Function: \textbf{jmethod} [\textbf{java}] \textit{class-ref method-name \&rest parameter-class-refs}

\begin{adjustwidth}{5em}{5em}
Returns a reference to the Java method METHOD-NAME of CLASS-REF with the given PARAMETER-CLASS-REFS.
\end{adjustwidth}

\paragraph{}
\label{JAVA:JMETHOD-LET}
\index{JMETHOD-LET}
--- Macro: \textbf{jmethod-let} [\textbf{java}] \textit{}

\begin{adjustwidth}{5em}{5em}
not-documented
\end{adjustwidth}

\paragraph{}
\label{JAVA:JMETHOD-NAME}
\index{JMETHOD-NAME}
--- Function: \textbf{jmethod-name} [\textbf{java}] \textit{method}

\begin{adjustwidth}{5em}{5em}
Returns the name of METHOD as a Lisp string
\end{adjustwidth}

\paragraph{}
\label{JAVA:JMETHOD-PARAMS}
\index{JMETHOD-PARAMS}
--- Function: \textbf{jmethod-params} [\textbf{java}] \textit{method}

\begin{adjustwidth}{5em}{5em}
Returns a vector of parameter types (Java classes) for METHOD
\end{adjustwidth}

\paragraph{}
\label{JAVA:JMETHOD-RETURN-TYPE}
\index{JMETHOD-RETURN-TYPE}
--- Function: \textbf{jmethod-return-type} [\textbf{java}] \textit{method}

\begin{adjustwidth}{5em}{5em}
Returns the result type (Java class) of the METHOD
\end{adjustwidth}

\paragraph{}
\label{JAVA:JNEW}
\index{JNEW}
--- Function: \textbf{jnew} [\textbf{java}] \textit{constructor \&rest args}

\begin{adjustwidth}{5em}{5em}
Invokes the Java constructor CONSTRUCTOR with the arguments ARGS.
\end{adjustwidth}

\paragraph{}
\label{JAVA:JNEW-ARRAY}
\index{JNEW-ARRAY}
--- Function: \textbf{jnew-array} [\textbf{java}] \textit{element-type \&rest dimensions}

\begin{adjustwidth}{5em}{5em}
Creates a new Java array of type ELEMENT-TYPE, with the given DIMENSIONS.
\end{adjustwidth}

\paragraph{}
\label{JAVA:JNEW-ARRAY-FROM-ARRAY}
\index{JNEW-ARRAY-FROM-ARRAY}
--- Function: \textbf{jnew-array-from-array} [\textbf{java}] \textit{element-type array}

\begin{adjustwidth}{5em}{5em}
Returns a new Java array with base type ELEMENT-TYPE (a string or a class-ref)
   initialized from ARRAY.
\end{adjustwidth}

\paragraph{}
\label{JAVA:JNEW-ARRAY-FROM-LIST}
\index{JNEW-ARRAY-FROM-LIST}
--- Function: \textbf{jnew-array-from-list} [\textbf{java}] \textit{element-type list}

\begin{adjustwidth}{5em}{5em}
Returns a new Java array with base type ELEMENT-TYPE (a string or a class-ref)
   initialized from a Lisp list.
\end{adjustwidth}

\paragraph{}
\label{JAVA:JNEW-RUNTIME-CLASS}
\index{JNEW-RUNTIME-CLASS}
--- Function: \textbf{jnew-runtime-class} [\textbf{java}] \textit{class-name \&rest args \&key (superclass java.lang.Object) interfaces constructors methods fields (access-flags (quote (public))) annotations}

\begin{adjustwidth}{5em}{5em}
Creates and loads a Java class with methods calling Lisp closures
   as given in METHODS.  CLASS-NAME and SUPER-NAME are strings,
   INTERFACES is a list of strings, CONSTRUCTORS, METHODS and FIELDS are
   lists of constructor, method and field definitions.

   Constructor definitions - currently NOT supported - are lists of the form
   (argument-types function \&optional super-invocation-arguments)
   where argument-types is a list of strings and function is a lisp function of
   (1+ (length argument-types)) arguments; the instance (`this') is passed in as
   the last argument. The optional super-invocation-arguments is a list of numbers
   between 1 and (length argument-types), where the number k stands for the kth argument
   to the just defined constructor. If present, the constructor of the superclass
   will be called with the appropriate arguments. E.g., if the constructor definition is
   (("java.lang.String" "int") \#'(lambda (string i this) ...) (2 1))
   then the constructor of the superclass with argument types (int, java.lang.String) will
   be called with the second and first arguments.

   Method definitions are lists of the form

     (METHOD-NAME RETURN-TYPE ARGUMENT-TYPES FUNCTION \&key MODIFIERS ANNOTATIONS)

   where 
      METHOD-NAME is a string 
      RETURN-TYPE denotes the type of the object returned by the method
      ARGUMENT-TYPES is a list of parameters to the method
      
        The types are either strings naming fully qualified java classes or Lisp keywords referring to 
        primitive types (:void, :int, etc.).

     FUNCTION is a Lisp function of minimum arity (1+ (length
     argument-types)). The instance (`this') is passed as the first
     argument.

   Field definitions are lists of the form (field-name type \&key modifiers annotations).
\end{adjustwidth}

\paragraph{}
\label{JAVA:JNULL-REF-P}
\index{JNULL-REF-P}
--- Function: \textbf{jnull-ref-p} [\textbf{java}] \textit{object}

\begin{adjustwidth}{5em}{5em}
Returns a non-NIL value when the JAVA-OBJECT `object` is `null`,
or signals a TYPE-ERROR condition if the object isn't of
the right type.
\end{adjustwidth}

\paragraph{}
\label{JAVA:JOBJECT-CLASS}
\index{JOBJECT-CLASS}
--- Function: \textbf{jobject-class} [\textbf{java}] \textit{obj}

\begin{adjustwidth}{5em}{5em}
Returns the Java class that OBJ belongs to
\end{adjustwidth}

\paragraph{}
\label{JAVA:JOBJECT-LISP-VALUE}
\index{JOBJECT-LISP-VALUE}
--- Function: \textbf{jobject-lisp-value} [\textbf{java}] \textit{java-object}

\begin{adjustwidth}{5em}{5em}
Attempts to coerce JAVA-OBJECT into a Lisp object.
\end{adjustwidth}

\paragraph{}
\label{JAVA:JPROPERTY-VALUE}
\index{JPROPERTY-VALUE}
--- Function: \textbf{jproperty-value} [\textbf{java}] \textit{object property}

\begin{adjustwidth}{5em}{5em}
setf-able access on the Java Beans notion of property named PROPETRY on OBJECT.
\end{adjustwidth}

\paragraph{}
\label{JAVA:JREGISTER-HANDLER}
\index{JREGISTER-HANDLER}
--- Function: \textbf{jregister-handler} [\textbf{java}] \textit{object event handler \&key data count}

\begin{adjustwidth}{5em}{5em}
not-documented
\end{adjustwidth}

\paragraph{}
\label{JAVA:JRESOLVE-METHOD}
\index{JRESOLVE-METHOD}
--- Function: \textbf{jresolve-method} [\textbf{java}] \textit{method-name instance \&rest args}

\begin{adjustwidth}{5em}{5em}
Finds the most specific Java method METHOD-NAME on INSTANCE applicable to arguments ARGS. Returns NIL if no suitable method is found. The algorithm used for resolution is the same used by JCALL when it is called with a string as the first parameter (METHOD-REF).
\end{adjustwidth}

\paragraph{}
\label{JAVA:JRUN-EXCEPTION-PROTECTED}
\index{JRUN-EXCEPTION-PROTECTED}
--- Function: \textbf{jrun-exception-protected} [\textbf{java}] \textit{closure}

\begin{adjustwidth}{5em}{5em}
Invokes the function CLOSURE and returns the result.  Signals an error if stack or heap exhaustion occurs.
\end{adjustwidth}

\paragraph{}
\label{JAVA:JSTATIC}
\index{JSTATIC}
--- Function: \textbf{jstatic} [\textbf{java}] \textit{method class \&rest args}

\begin{adjustwidth}{5em}{5em}
Invokes the static method METHOD on class CLASS with ARGS.
\end{adjustwidth}

\paragraph{}
\label{JAVA:JSTATIC-RAW}
\index{JSTATIC-RAW}
--- Function: \textbf{jstatic-raw} [\textbf{java}] \textit{method class \&rest args}

\begin{adjustwidth}{5em}{5em}
Invokes the static method METHOD on class CLASS with ARGS. Does not attempt to coerce the arguments or result into a Lisp object.
\end{adjustwidth}

\paragraph{}
\label{JAVA:MAKE-CLASSLOADER}
\index{MAKE-CLASSLOADER}
--- Function: \textbf{make-classloader} [\textbf{java}] \textit{\&optional parent}

\begin{adjustwidth}{5em}{5em}
not-documented
\end{adjustwidth}

\paragraph{}
\label{JAVA:MAKE-IMMEDIATE-OBJECT}
\index{MAKE-IMMEDIATE-OBJECT}
--- Function: \textbf{make-immediate-object} [\textbf{java}] \textit{object \&optional type}

\begin{adjustwidth}{5em}{5em}
Attempts to coerce a given Lisp object into a java-object of the
given type.  If type is not provided, works as jobject-lisp-value.
Currently, type may be :BOOLEAN, treating the object as a truth value,
or :REF, which returns Java null if NIL is provided.

Deprecated.  Please use JAVA:+NULL+, JAVA:+TRUE+, and JAVA:+FALSE+ for
constructing wrapped primitive types, JAVA:JOBJECT-LISP-VALUE for converting a
JAVA:JAVA-OBJECT to a Lisp value, or JAVA:JNULL-REF-P to distinguish a wrapped
null JAVA-OBJECT from NIL.
\end{adjustwidth}

\paragraph{}
\label{JAVA:REGISTER-JAVA-EXCEPTION}
\index{REGISTER-JAVA-EXCEPTION}
--- Function: \textbf{register-java-exception} [\textbf{java}] \textit{exception-name condition-symbol}

\begin{adjustwidth}{5em}{5em}
Registers the Java Throwable named by the symbol EXCEPTION-NAME as the condition designated by CONDITION-SYMBOL.  Returns T if successful, NIL if not.
\end{adjustwidth}

\paragraph{}
\label{JAVA:UNREGISTER-JAVA-EXCEPTION}
\index{UNREGISTER-JAVA-EXCEPTION}
--- Function: \textbf{unregister-java-exception} [\textbf{java}] \textit{exception-name}

\begin{adjustwidth}{5em}{5em}
Unregisters the Java Throwable EXCEPTION-NAME previously registered by REGISTER-JAVA-EXCEPTION.
\end{adjustwidth}



\section{THREADS}

Multithreading

\subsection{API}

% include autogen docs for the THREADS package.
\paragraph{}
\label{THREADS:CURRENT-THREAD}
\index{CURRENT-THREAD}
--- Function: \textbf{current-thread} [\textbf{threads}] \textit{}

\begin{adjustwidth}{5em}{5em}
Returns a reference to invoking thread.
\end{adjustwidth}

\paragraph{}
\label{THREADS:DESTROY-THREAD}
\index{DESTROY-THREAD}
--- Function: \textbf{destroy-thread} [\textbf{threads}] \textit{}

\begin{adjustwidth}{5em}{5em}
not-documented
\end{adjustwidth}

\paragraph{}
\label{THREADS:GET-MUTEX}
\index{GET-MUTEX}
--- Function: \textbf{get-mutex} [\textbf{threads}] \textit{mutex}

\begin{adjustwidth}{5em}{5em}
Acquires a lock on the `mutex'.
\end{adjustwidth}

\paragraph{}
\label{THREADS:INTERRUPT-THREAD}
\index{INTERRUPT-THREAD}
--- Function: \textbf{interrupt-thread} [\textbf{threads}] \textit{thread function \&rest args}

\begin{adjustwidth}{5em}{5em}
Interrupts THREAD and forces it to apply FUNCTION to ARGS.
When the function returns, the thread's original computation continues. If  multiple interrupts are queued for a thread, they are all run, but the order is not guaranteed.
\end{adjustwidth}

\paragraph{}
\label{THREADS:MAILBOX-EMPTY-P}
\index{MAILBOX-EMPTY-P}
--- Function: \textbf{mailbox-empty-p} [\textbf{threads}] \textit{mailbox}

\begin{adjustwidth}{5em}{5em}
Returns non-NIL if the mailbox can be read from, NIL otherwise.
\end{adjustwidth}

\paragraph{}
\label{THREADS:MAILBOX-PEEK}
\index{MAILBOX-PEEK}
--- Function: \textbf{mailbox-peek} [\textbf{threads}] \textit{mailbox}

\begin{adjustwidth}{5em}{5em}
Returns two values. The second returns non-NIL when the mailbox
is empty. The first is the next item to be read from the mailbox.

Note that due to multi-threading, the first value returned upon
peek, may be different from the one returned upon next read in the
calling thread.
\end{adjustwidth}

\paragraph{}
\label{THREADS:MAILBOX-READ}
\index{MAILBOX-READ}
--- Function: \textbf{mailbox-read} [\textbf{threads}] \textit{mailbox}

\begin{adjustwidth}{5em}{5em}
Blocks on the mailbox until an item is available for reading.
When an item is available, it is returned.
\end{adjustwidth}

\paragraph{}
\label{THREADS:MAILBOX-SEND}
\index{MAILBOX-SEND}
--- Function: \textbf{mailbox-send} [\textbf{threads}] \textit{mailbox item}

\begin{adjustwidth}{5em}{5em}
Sends an item into the mailbox, notifying 1 waiter
to wake up for retrieval of that object.
\end{adjustwidth}

\paragraph{}
\label{THREADS:MAKE-MAILBOX}
\index{MAKE-MAILBOX}
--- Function: \textbf{make-mailbox} [\textbf{threads}] \textit{\&key ((queue g282154) NIL)}

\begin{adjustwidth}{5em}{5em}
not-documented
\end{adjustwidth}

\paragraph{}
\label{THREADS:MAKE-MUTEX}
\index{MAKE-MUTEX}
--- Function: \textbf{make-mutex} [\textbf{threads}] \textit{\&key ((in-use g282414) NIL)}

\begin{adjustwidth}{5em}{5em}
not-documented
\end{adjustwidth}

\paragraph{}
\label{THREADS:MAKE-THREAD}
\index{MAKE-THREAD}
--- Function: \textbf{make-thread} [\textbf{threads}] \textit{function \&key name}

\begin{adjustwidth}{5em}{5em}
not-documented
\end{adjustwidth}

\paragraph{}
\label{THREADS:MAKE-THREAD-LOCK}
\index{MAKE-THREAD-LOCK}
--- Function: \textbf{make-thread-lock} [\textbf{threads}] \textit{}

\begin{adjustwidth}{5em}{5em}
Returns an object to be used with the `with-thread-lock' macro.
\end{adjustwidth}

\paragraph{}
\label{THREADS:MAPCAR-THREADS}
\index{MAPCAR-THREADS}
--- Function: \textbf{mapcar-threads} [\textbf{threads}] \textit{}

\begin{adjustwidth}{5em}{5em}
not-documented
\end{adjustwidth}

\paragraph{}
\label{THREADS:OBJECT-NOTIFY}
\index{OBJECT-NOTIFY}
--- Function: \textbf{object-notify} [\textbf{threads}] \textit{object}

\begin{adjustwidth}{5em}{5em}
Wakes up a single thread that is waiting on OBJECT's monitor.
If any threads are waiting on this object, one of them is chosen to be awakened. The choice is arbitrary and occurs at the discretion of the implementation. A thread waits on an object's monitor by calling one of the wait methods.
\end{adjustwidth}

\paragraph{}
\label{THREADS:OBJECT-NOTIFY-ALL}
\index{OBJECT-NOTIFY-ALL}
--- Function: \textbf{object-notify-all} [\textbf{threads}] \textit{object}

\begin{adjustwidth}{5em}{5em}
Wakes up all threads that are waiting on this OBJECT's monitor.
A thread waits on an object's monitor by calling one of the wait methods.
\end{adjustwidth}

\paragraph{}
\label{THREADS:OBJECT-WAIT}
\index{OBJECT-WAIT}
--- Function: \textbf{object-wait} [\textbf{threads}] \textit{object \&optional timeout}

\begin{adjustwidth}{5em}{5em}
Causes the current thread to block until object-notify or object-notify-all is called on OBJECT.
Optionally unblock execution after TIMEOUT seconds.  A TIMEOUT of zero
means to wait indefinitely.
A non-zero TIMEOUT of less than a nanosecond is interpolated as a nanosecond wait.
See the documentation of java.lang.Object.wait() for further
information.

\end{adjustwidth}

\paragraph{}
\label{THREADS:RELEASE-MUTEX}
\index{RELEASE-MUTEX}
--- Function: \textbf{release-mutex} [\textbf{threads}] \textit{mutex}

\begin{adjustwidth}{5em}{5em}
Releases a lock on the `mutex'.
\end{adjustwidth}

\paragraph{}
\label{THREADS:SYNCHRONIZED-ON}
\index{SYNCHRONIZED-ON}
--- Special Operator: \textbf{synchronized-on} [\textbf{threads}] \textit{}

\begin{adjustwidth}{5em}{5em}
not-documented
\end{adjustwidth}

\paragraph{}
\label{THREADS:THREAD}
\index{THREAD}
--- Class: \textbf{thread} [\textbf{threads}] \textit{}

\begin{adjustwidth}{5em}{5em}
not-documented
\end{adjustwidth}

\paragraph{}
\label{THREADS:THREAD-ALIVE-P}
\index{THREAD-ALIVE-P}
--- Function: \textbf{thread-alive-p} [\textbf{threads}] \textit{thread}

\begin{adjustwidth}{5em}{5em}
Boolean predicate whether THREAD is alive.
\end{adjustwidth}

\paragraph{}
\label{THREADS:THREAD-JOIN}
\index{THREAD-JOIN}
--- Function: \textbf{thread-join} [\textbf{threads}] \textit{thread}

\begin{adjustwidth}{5em}{5em}
Waits for thread to finish.
\end{adjustwidth}

\paragraph{}
\label{THREADS:THREAD-NAME}
\index{THREAD-NAME}
--- Function: \textbf{thread-name} [\textbf{threads}] \textit{}

\begin{adjustwidth}{5em}{5em}
not-documented
\end{adjustwidth}

\paragraph{}
\label{THREADS:THREADP}
\index{THREADP}
--- Function: \textbf{threadp} [\textbf{threads}] \textit{}

\begin{adjustwidth}{5em}{5em}
not-documented
\end{adjustwidth}

\paragraph{}
\label{THREADS:WITH-MUTEX}
\index{WITH-MUTEX}
--- Macro: \textbf{with-mutex} [\textbf{threads}] \textit{}

\begin{adjustwidth}{5em}{5em}
not-documented
\end{adjustwidth}

\paragraph{}
\label{THREADS:WITH-THREAD-LOCK}
\index{WITH-THREAD-LOCK}
--- Macro: \textbf{with-thread-lock} [\textbf{threads}] \textit{}

\begin{adjustwidth}{5em}{5em}
not-documented
\end{adjustwidth}

\paragraph{}
\label{THREADS:YIELD}
\index{YIELD}
--- Function: \textbf{yield} [\textbf{threads}] \textit{}

\begin{adjustwidth}{5em}{5em}
A hint to the scheduler that the current thread is willing to yield its current use of a processor. The scheduler is free to ignore this hint. 

See java.lang.Thread.yield().
\end{adjustwidth}



\section{EXTENSIONS}

The symbols in the EXTENSIONS package (nicknamed ``EXT'') constitutes
extensions to the ANSI standard that are potentially useful to the
user.  They include functions for manipulating network sockets,
running external programs, registering object finalizers, constructing
reference weakly held by the garbage collector and others.

See \ref{Extensible Sequences} for a generic function interface to
the native JVM contract for \code{java.util.List}.

\subsection{API}

% include autogen docs for the EXTENSIONS package.
%CADDR
  Function: (not documented)
%CADR
  Function: (not documented)
%CAR
  Function: (not documented)
%CDR
  Function: (not documented)
*AUTOLOAD-VERBOSE*
  Variable: (not documented)
*BATCH-MODE*
  Variable: (not documented)
*COMMAND-LINE-ARGUMENT-LIST*
  Variable: (not documented)
*DEBUG-CONDITION*
  Variable: (not documented)
*DEBUG-LEVEL*
  Variable: (not documented)
*DISASSEMBLER*
  Variable: (not documented)
*ED-FUNCTIONS*
  Variable: (not documented)
*ENABLE-INLINE-EXPANSION*
  Variable: (not documented)
*INSPECTOR-HOOK*
  Variable: (not documented)
*LISP-HOME*
  Variable: (not documented)
*LOAD-TRUENAME-FASL*
  Variable: (not documented)
*PRINT-STRUCTURE*
  Variable: (not documented)
*REQUIRE-STACK-FRAME*
  Variable: (not documented)
*SAVED-BACKTRACE*
  Variable: (not documented)
*SUPPRESS-COMPILER-WARNINGS*
  Variable: (not documented)
*WARN-ON-REDEFINITION*
  Variable: (not documented)
ADJOIN-EQL
  Function: (not documented)
ARGLIST
  Function: (not documented)
ASSQ
  Function: (not documented)
ASSQL
  Function: (not documented)
AUTOLOAD
  Function: (not documented)
AUTOLOAD-MACRO
  Function: (not documented)
AUTOLOADP
  Function: (not documented)
AVER
  Function: (not documented)
CANCEL-FINALIZATION
  Function: (not documented)
CHAR-TO-UTF8
  Function: (not documented)
CHARPOS
  Function: (not documented)
CLASSP
  Function: (not documented)
COLLECT
  Function: (not documented)
COMPILE-FILE-IF-NEEDED
  Function: (not documented)
COMPILE-SYSTEM
  Function: (not documented)
COMPILER-ERROR
  Function: (not documented)
  Class: (not documented)
COMPILER-UNSUPPORTED-FEATURE-ERROR
  Class: (not documented)
DESCRIBE-COMPILER-POLICY
  Function: (not documented)
DOUBLE-FLOAT-NEGATIVE-INFINITY
  Variable: (not documented)
DOUBLE-FLOAT-POSITIVE-INFINITY
  Variable: (not documented)
DUMP-JAVA-STACK
  Function: (not documented)
EXIT
  Function: (not documented)
FEATUREP
  Function: (not documented)
FILE-DIRECTORY-P
  Function: (not documented)
FINALIZE
  Function: (not documented)
FIXNUMP
  Function: (not documented)
GC
  Function: (not documented)
GET-FLOATING-POINT-MODES
  Function: (not documented)
GET-SOCKET-STREAM
  Function: :ELEMENT-TYPE must be CHARACTER or (UNSIGNED-BYTE 8); the default is CHARACTER.
GETENV
  Function: Return the value of the environment VARIABLE if it exists, otherwise return NIL.
GROVEL-JAVA-DEFINITIONS
  Function: (not documented)
INIT-GUI
  Function: (not documented)
INTERNAL-COMPILER-ERROR
  Function: (not documented)
  Class: (not documented)
INTERRUPT-LISP
  Function: (not documented)
JAR-PATHNAME
  Class: (not documented)
MACROEXPAND-ALL
  Function: (not documented)
MAILBOX
  Class: (not documented)
MAKE-DIALOG-PROMPT-STREAM
  Function: (not documented)
MAKE-SERVER-SOCKET
  Function: (not documented)
MAKE-SLIME-INPUT-STREAM
  Function: (not documented)
MAKE-SLIME-OUTPUT-STREAM
  Function: (not documented)
MAKE-SOCKET
  Function: (not documented)
MAKE-TEMP-FILE
  Function: (not documented)
MAKE-WEAK-REFERENCE
  Function: (not documented)
MEMQ
  Function: (not documented)
MEMQL
  Function: (not documented)
MOST-NEGATIVE-JAVA-LONG
  Variable: (not documented)
MOST-POSITIVE-JAVA-LONG
  Variable: (not documented)
MUTEX
  Class: (not documented)
NEQ
  Function: (not documented)
NIL-VECTOR
  Class: (not documented)
PATHNAME-JAR-P
  Function: Predicate for whether PATHNAME references a JAR.
PATHNAME-URL-P
  Function: Predicate for whether PATHNAME references a URL.
PRECOMPILE
  Function: (not documented)
PROBE-DIRECTORY
  Function: (not documented)
PROCESS
  Function: (not documented)
PROCESS-ALIVE-P
  Function: (not documented)
PROCESS-ERROR
  Function: (not documented)
PROCESS-EXIT-CODE
  Function: (not documented)
PROCESS-INPUT
  Function: (not documented)
PROCESS-KILL
  Function: (not documented)
PROCESS-OUTPUT
  Function: (not documented)
PROCESS-P
  Function: (not documented)
PROCESS-WAIT
  Function: (not documented)
QUIT
  Function: (not documented)
RESOLVE
  Function: (not documented)
RUN-PROGRAM
  Function: (not documented)
RUN-SHELL-COMMAND
  Function: (not documented)
SERVER-SOCKET-CLOSE
  Function: (not documented)
SET-FLOATING-POINT-MODES
  Function: (not documented)
SHOW-RESTARTS
  Function: (not documented)
SIMPLE-SEARCH
  Function: (not documented)
SIMPLE-STRING-FILL
  Function: (not documented)
SIMPLE-STRING-SEARCH
  Function: (not documented)
SINGLE-FLOAT-NEGATIVE-INFINITY
  Variable: (not documented)
SINGLE-FLOAT-POSITIVE-INFINITY
  Variable: (not documented)
SLIME-INPUT-STREAM
  Class: (not documented)
SLIME-OUTPUT-STREAM
  Class: (not documented)
SOCKET-ACCEPT
  Function: (not documented)
SOCKET-CLOSE
  Function: (not documented)
SOCKET-LOCAL-ADDRESS
  Function: Returns the local address of the given socket as a dotted quad string.
SOCKET-LOCAL-PORT
  Function: Returns the local port number of the given socket.
SOCKET-PEER-ADDRESS
  Function: Returns the peer address of the given socket as a dotted quad string.
SOCKET-PEER-PORT
  Function: Returns the peer port number of the given socket.
SOURCE
  Function: (not documented)
SOURCE-FILE-POSITION
  Function: (not documented)
SOURCE-PATHNAME
  Function: (not documented)
SPECIAL-VARIABLE-P
  Function: (not documented)
STRING-FIND
  Function: (not documented)
STRING-INPUT-STREAM-CURRENT
  Function: (not documented)
STRING-POSITION
  Function: (not documented)
STYLE-WARN
  Function: (not documented)
TRULY-THE
  Function: (not documented)
UPTIME
  Function: (not documented)
URI-DECODE
  Function: (not documented)
URI-ENCODE
  Function: (not documented)
URL-PATHNAME
  Class: (not documented)
URL-PATHNAME-AUTHORITY
  Function: (not documented)
URL-PATHNAME-FRAGMENT
  Function: (not documented)
URL-PATHNAME-QUERY
  Function: (not documented)
URL-PATHNAME-SCHEME
  Function: (not documented)
WEAK-REFERENCE
  Class: (not documented)
WEAK-REFERENCE-VALUE
  Function: (not documented)


\chapter{Beyond ANSI}

Naturally, in striving to be a useful contemporary Common Lisp
implementation, ABCL endeavors to include extensions beyond the ANSI
specification which are either widely adopted or are especially useful
in working with the hosting JVM.

\section{Implementation Dependent}
\begin{enumerate}
  \item Compiler to JVM 5 bytecode
  \item Pathname extensions
\end{enumerate}

\section{Pathname}

We implment an extension to the Pathname that allows for the
description and retrieval of resources named in a URI scheme that the
JVM ``understands''.  Support is built-in to the ``http'' and
``https'' implementations but additional protocol handlers may be
installed at runtime by having JVM symbols present in the
sun.net.protocol.dynmamic pacakge. See [JAVA2006] for more details.

ABCL has created specializations of the ANSI Pathname object to
enable to use of URIs to address dynamically loaded resources for the
JVM.  A URL-PATHNAME has a corresponding URL whose cannoical
representation is defined to be the NAMESTRING of the Pathname.

PATHNAME : URL-PATHNAME : JAR-PATHNAME

Both URL-PATHNAME and JAR-PATHNAME may be used anu where will a
PATHNAME is accepted witht the following caveats

A stream obtained via OPEN on a URL-PATHNAME cannot be the target of
write operations.

No canonicalization is performed on the underlying URI (i.e. the
implementation does not attempt to compute the current name of the
representing resource unless it is requested to be resolved.)  Upon
resolution, any cannoicalization procedures followed in resolving the
resource (e.g. following redirects) are discarded.  

The implementation of URL-PATHNAME allows the ABCL user to laod dynamically
code from the network.  For example, for Quicklisp.

\begin{listing-lisp}
  CL-USER> (load "http://beta.quicklisp.org/quicklisp.lisp")
\end{listing-lisp}

will load and execute the Quicklisp setup code.

\ref{XACH2011}
         
\section{Extensible Sequences}

\ref{RHODES2007}

The SEQUENCE package fully implements Christopher Rhodes' proposal for
extensible sequences.  These user extensible sequences are used
directly in \code{java-collections.lisp} provide these CLOS
abstractions on the standard Java collection classes as defined by the
\code{java.util.List} contract.

This extension is not automatically loaded by the implementation.   It
may be loaded via:

\begin{listing-lisp}
CL-USER> (require 'java-collections)
\end{listing-lisp}

if both extensible sequences and their application to Java collections
is required, or

\begin{listing-lisp}
CL-USER> (require 'extensible-sequences)
\end{listing-lisp}

if only the extensible sequences API as specified in \ref{RHODES2007} is
required.

Note that \code{(require 'java-collections)} must be issued before
\code{java.util.List} or any subclass is used as a specializer in a CLOS
method definition (see the section below).

\section{Extensions to CLOS}

There is an additional syntax for specializing the parameter of a
generic function on a java class, viz. \code{(java:jclass CLASS-STRING)}
where \code{CLASS-STRING} is a string naming a Java class in dotted package
form.

For instance the following specialization would perhaps allow one to
print more information about the contents of a java.util.Collection
object

\begin{listing-lisp}
(defmethod print-object ((coll (java:jclass "java.util.Collection"))
                         stream)
  ;;; ...
)
\end{listing-lisp}

If the class had been loaded via a classloader other than the original
the class you wish to specialize on, one needs to specify the
classloader as an optional third argument.

\begin{listing-lisp}

(defparameter *other-classloader*
  (jcall "getBaseLoader" cl-user::*classpath-manager*))
  
(defmethod print-object ((device-id (java:jclass "dto.nbi.service.hdm.alcatel.com.NBIDeviceID" *other-classloader*))
                         stream)
  ;;; ...
)
\end{listing-lisp}

\section{Extensions to the Reader}

We implement a special hexadecimal escape sequence for specifying
characters to the Lisp reader, namely we allow a sequences of the form
\# \textbackslash Uxxxx to be processed by the reader as character whose code is
specified by the hexadecimal digits ``xxxx''.  The hexadecimal sequence
must be exactly four digits long, padded by leading zeros for values
less than 0x1000.

Note that this sequence is never output by the implementation.  Instead,
the corresponding Unicode character is output for characters whose
code is greater than 0x00ff.

\section{ASDF}

asdf-2.017 is packaged as core component of ABCL.  By default, ASDF is
not loaded, as it relies on the CLOS subsystem which can take a bit of
time to initialize.

\begin{listing-lisp}
CL-USER> (require 'asdf)
\end{listing-lisp}

\chapter{Contrib}

\section{abcl-asdf}

Allow ASDF system definition which dynamically loads JVM artifacts
such as jar archives via a Maven encapsulation.

ASDF components added:  JAR-FILE, JAR-DIRECTORY, CLASS-FILE-DIRECTORY
and MVN.

\section{asdf-install}

An implementation of ASDF-INSTALL.  Superceded by Quicklisp (qv.)

\section{asdf-jar}

ASDF-JAR provides a system for packaging ASDF systems into jar
archives for ABCL.  Given a running ABCL image with loadable ASDF
systems the code in this package will recursively package all the
required source and fasls in a jar archive.

\section{jss}

Java Syntax sucks, so we introduce the \#" macro.


\chapter{History}

ABCL was originally the extension language for the J editor, which was
started in 1998 by Peter Graves.  Sometime in 2003, it seems that a
lot of code that had previously not been released publically was
suddenly committed that enabled ABCL to be plausibly termed an ANSI
Common Lisp implementation.

In 2006, the implementation was transferred to the current
maintainers, who have strived to improve its usability as a
contemporary Common Lisp implementation.

In 201x, with the publication of this Manual explicitly stating the
conformance of Armed Bear Common Lisp to ANSI, we release abcl-1.0.




\section{References}

[Java2000]:  A New Era for Java Protocol Handlers.
\url{http://java.sun.com/developer/onlineTraining/protocolhandlers/}

[Xach2011]:  Quicklisp:  A system for quickly constructing Common Lisp
libraries.  \url{http://www.quicklisp.org/}


\end{document}

% TODO
%   1.  Create mechanism for swigging DocString and Lisp docs into
%       sections.

